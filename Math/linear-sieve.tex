\subsection{线性筛}
lpf 为最小质因子的标号;
mu 为莫比乌斯函数;
phi 为欧拉函数;
e 为质因子最高次幂,d 为因数个数;
f 为因数和,g 为最小质因子的幂和,即 $p + p^{1} + p^{2} + \cdots + p^{k}$ 。
理论上积性函数都可以线性筛。

\begin{lstlisting}
const int MAXN = 1e7 + 5;
int prime[MAXN / 15], prime_cnt;
int lpf[MAXN], e[MAXN], d[MAXN], mu[MAXN], phi[MAXN];
void sieve() {
  prime[lpf[1] = 0] = 1, e[1] = 0, d[1] = 1, mu[1] = 1, phi[1] = 1;
  for (int i = 2; i < MAXN; ++i) {
    if (!lpf[i]) {
      prime[lpf[i] = ++prime_cnt] = i;
      mu[i] = -1, phi[i] = i - 1;
      e[i] = 1, d[i] = 2;
      g[i] = f[i] = i + 1;
    }
    for (int j = 1, x; j <= lpf[i] && (x = i * prime[j]) < MAXN; ++j) {
      lpf[x] = j;
      if (j < lpf[i]) {
        mu[x] = -mu[i], phi[x] = phi[i] * (prime[j] - 1);
        e[x] = 1, d[x] = d[i] * 2;
        g[x] = 1 + prime[j], f[x] = f[i] * f[prime[j]];
      } else { // i % prime[j] == 0
        mu[x] = 0, phi[x] = phi[i] * prime[j];
        e[x] = e[i] + 1, d[x] = d[i] / e[x] * (e[x] + 1);
        g[x] = g[i] * prime[j] + 1, f[x] = f[i] / g[i] * g[x];
      }
    }
  }
}
\end{lstlisting}