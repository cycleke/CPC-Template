\subsection{BGSG}
大步小步算法用来求离散对数 $x^{k} \equiv a \pmod p$。
求解 $a^{x} \equiv b \pmod p$ 则,
令 $x = A\lceil\sqrt{p}\rceil - B, 0 \le A, B \le \lceil\sqrt{p}\rceil$,
有 $a^{A\lceil\sqrt{p}\rceil} \equiv b a^{B} \pmod p$,
先枚举 $A$ 之后在哈希表中查找 $B$ 就行。
\begin{lstlisting}
// Finds the primitive root modulo p
int generator(int p) {
  vector<int> fact;
  int phi = p - 1, n = phi;
  for (int i = 2; i * i <= n; ++i) {
    if (n % i == 0) {
      fact.push_back(i);
      while (n % i == 0) n /= i;
    }
  }
  if (n > 1) fact.push_back(n);
  for (int res = 2; res <= p; ++res) {
    bool ok = true;
    for (int factor : fact)
      if (mpow(res, phi / factor, p) == 1) {
        ok = false;
        break;
      }
    if (ok) return res;
  }
  return -1;
}
vector<int> BSGS(int n, int k, int a) {
  if (a == 0) return vector<int>({0});

  int g = generator(n);
  // Baby-step giant-step discrete logarithm algorithm
  int sq = (int)sqrt(n + .0) + 1;
  vector<pair<int, int>> dec(sq);
  for (int i = 1; i <= sq; ++i)
    dec[i - 1] = {mpow(g, i * sq * k % (n - 1), n), i};

  sort(dec.begin(), dec.end());
  int any_ans = -1;
  for (int i = 0; i < sq; ++i) {
    int my = mpow(g, i * k % (n - 1), n) * a % n;
    auto it = lower_bound(dec.begin(), dec.end(), make_pair(my, 0));
    if (it != dec.end() && it->first == my) {
      any_ans = it->second * sq - i;
      break;
    }
  }
  if (any_ans == -1) return vector<int>();
  // Print all possible answers
  int delta = (n - 1) / __gcd(k, n - 1);
  vector<int> ans;
  for (int cur = any_ans % delta; cur < n - 1; cur += delta)
    ans.push_back(mpow(g, cur, n));
  sort(ans.begin(), ans.end());
  return ans;
}
\end{lstlisting}
