\subsection{中国剩余定理}
\begin{lstlisting}
ll inv(ll a, ll p) {
  ll x, y;
  exgcd(a, p, x, y);
  return (x + p) % p;
}
ll CRT(ll n, ll *a, ll *m) {
  ll lcm = 1, res = 0;
  for (ll i = 0; i < n; ++i) lcm *= m[i];
  for (ll i = 0; i < n; ++i) {
    ll t = lcm / m[i], x = inv(t, m[i]);
    res = (res + a[i] * t % lcm * x) % lcm;
  }
  return res;
}
\end{lstlisting}

模数不互质的情况
$
  \begin{cases}
    x \equiv a_{1} &(\pmod m_{1}) \\
    x \equiv a_{2} &(\pmod m_{2})
  \end{cases}
$,
则转换为$m_{1}p - m_{2}q = a_{2} - a_{1}$,
最终解(若有解)为$x \equiv m_{1}p + a_{1} \pmod{\lcm(m_{1}, m_{2})}$。

拓展中国剩余定理:
\begin{lstlisting}
int exctr(int n, int *a, int *m) {
  int M = m[0], res = a[0];
  for (int i = 1; i < n; ++i) {
    int a = M, b = m[i], c = (a[i] - res % b + b) % b, x, y;
    int g = exgcd(a, b, x, y), bg = b / g;
    if (c % g != 0) return -1;
    x = 1LL * x * (c / g) % bg;
    res += x * M;
    M *= bg;
    res = (res % M + M) % M;
  }
  return res;
}
\end{lstlisting}