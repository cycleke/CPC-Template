\subsection{I/O 优化}
\begin{lstlisting}
namespace FastIO {
struct Control {
  int ct, val;
  Control(int Ct, int Val = -1) : ct(Ct), val(Val) {}
  inline Control operator()(int Val) { return Control(ct, Val); }
} _endl(0), _prs(1), _setprecision(2);

const int IO_SIZE = 1 << 16 | 127;

struct FastIO {
  char in[IO_SIZE], *p, *pp, out[IO_SIZE], *q, *qq, ch[20], *t, b, K, prs;
  FastIO() : p(in), pp(in), q(out), qq(out + IO_SIZE), t(ch), b(1), K(6) {}
  ~FastIO() { fwrite(out, 1, q - out, stdout); }
  inline char getc() {
    return p == pp && (pp = (p = in) + fread(in, 1, IO_SIZE, stdin), p == pp)
               ? (b = 0, EOF)
               : *p++;
  }
  inline void putc(char x) {
    q == qq && (fwrite(out, 1, q - out, stdout), q = out), *q++ = x;
  }
  inline void puts(const char str[]) {
    fwrite(out, 1, q - out, stdout), fwrite(str, 1, strlen(str), stdout),
        q = out;
  }
  inline void getline(string &s) {
    s = "";
    for (char ch; (ch = getc()) != '\n' && b;) s += ch;
  }
#define indef(T)                                                               \
  inline FastIO &operator>>(T &x) {                                            \
    x = 0;                                                                     \
    char f = 0, ch;                                                            \
    while (!isdigit(ch = getc()) && b) f |= ch == '-';                         \
    while (isdigit(ch)) x = (x << 1) + (x << 3) + (ch ^ 48), ch = getc();      \
    return x = f ? -x : x, *this;                                              \
  }
  indef(int);
  indef(long long);

  inline FastIO &operator>>(string &s) {
    s = "";
    char ch;
    while (isspace(ch = getc()) && b) {}
    while (!isspace(ch) && b) s += ch, ch = getc();
    return *this;
  }
  inline FastIO &operator>>(double &x) {
    x = 0;
    char f = 0, ch;
    double d = 0.1;
    while (!isdigit(ch = getc()) && b) f |= (ch == '-');
    while (isdigit(ch)) x = x * 10 + (ch ^ 48), ch = getc();
    if (ch == '.')
      while (isdigit(ch = getc())) x += d * (ch ^ 48), d *= 0.1;
    return x = f ? -x : x, *this;
  }
#define outdef(_T)                                                             \
  inline FastIO &operator<<(_T x) {                                            \
    !x && (putc('0'), 0), x < 0 && (putc('-'), x = -x);                        \
    while (x) *t++ = x % 10 + 48, x /= 10;                                     \
    while (t != ch) *q++ = *--t;                                               \
    return *this;                                                              \
  }
  outdef(int);
  outdef(long long);
  inline FastIO &operator<<(char ch) { return putc(ch), *this; }
  inline FastIO &operator<<(const char str[]) { return puts(str), *this; }
  inline FastIO &operator<<(const string &s) { return puts(s.c_str()), *this; }
  inline FastIO &operator<<(double x) {
    int k = 0;
    this->operator<<(int(x));
    putc('.');
    x -= int(x);
    prs && (x += 5 * pow(10, -K - 1));
    while (k < K) putc(int(x *= 10) ^ 48), x -= int(x), ++k;
    return *this;
  }
  inline FastIO &operator<<(const Control &cl) {
    switch (cl.ct) {
    case 0: putc('\n'); break;
    case 1: prs = cl.val; break;
    case 2: K = cl.val; break;
    }
    return *this;
  }
  inline operator bool() { return b; }
};
} // namespace FastIO
\end{lstlisting}
