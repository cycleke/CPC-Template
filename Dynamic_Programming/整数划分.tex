\subsection{整数划分}

$f_{i, j}$ 表示选了 $i$ 种不同的数字,和为 $j$ 的方案数。
$f_{i, j} = f_{i - 1, j - 1} + f_{i, j - i}$,
意义:要么新选一个 1,要么前面的数都加一。
若每个数字最多一个,$f_{i, j} = f_{i - 1, j - i} + f_{i, j - i}$。

求将 $n$ 划分为若干整数的方案,则设 $g_{n}$ 为答案,
代码如下,时间复杂度 $O(n\sqrt{n})$。
\begin{lstlisting}
int f[732], g[200001];
void init() {
  f[1] = 1, f[2] = 2, f[3] = 5, f[4] = 7;
  for (int i = 5; i < 732; ++i) f[i] = 3 + 2 * f[i - 2] - f[i - 4];
  for (int i = g[0] = 1; i <= n; ++i)
    for (int j = 1; f[j] <= i; ++j)
      g[i] = ((j + 1) >> 1 & 1) ? (g[i] + g[i - f[j]]) % MOD : (g[i] - g[i - f[j]] + MOD) % MOD;
}
\end{lstlisting}